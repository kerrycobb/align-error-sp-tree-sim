
\embedHeightFigure{0.78}{../../../sims/plots-tex/cropped-grid-theta-500bp.pdf}{
    \footnotesize
    \figtitle{Accuracy and precision of estimates of effective population sizes of the
    descendant branches of the tree scaled by the mutation rate ($\tippopsize\mu$)
    with 500 base pair loci.} \figlegend
    }{fig:theta500}
    



% \embedHeightFigure{0.78}{../../../sims/plots-tex/cropped-grid-theta-500bp.pdf}{
% \footnotesize
% \uline{Accuracy and precision of estimates of effective population sizes of the
% descendant branches of the tree scaled by the mutation rate ($\tippopsize\mu$)
% with 500 base pair loci.}
% Consistent with \mainfigsp: The left column shows estimates from \beast.
% The center column shows estimates from \ecoevolity using all sites and the
% right column shows estimates from \ecoevolity using a single SNP per locus.
% The top row are estimates from 200 data sets simulated without
% character-pattern errors.
% Rows labelled 20\% and 40\% singleton errors are estimates from the same
% alignments after singleton site patterns were changed to invariant sites with
% probabilities 0.2 and 0.4, respectively.
% Rows labelled 20\% and 40\% het errors are estimates from the same alignments
% after one copy of randomly paired gene copies within each species was replaced
% with the other with probabilities 0.2 and 0.4, respectively.
% Each plotted circle and associated error bars represent the posterior mean and 95\%
% credible interval.
% Circles and error bars are colored yellow if the effective sample size (ESS) of
% the estimate was less than 200, red if the potential scale reduction factor
% (PSRF) was greater than 1.2, and green if both conditions were true.
% The root mean square error (RMSE) and rate of poor MCMC behavior (RPMB) is
% given for each plot, the latter of which is the proportion of estimates with
% ESS < 200 and/or PSRF > 1.2.
% We generated the plots using matplotlib Version 3.1.1 \citep{matplotlib}.
% }{fig:theta500}
